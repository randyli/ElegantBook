
\chapter{布拉格石人}
16世纪的时候,哈布斯堡王朝几乎控制着中欧的大部分地区,包括荷兰,西班牙以及西班牙在美洲的殖民地。
哈布斯堡王朝是第一个真正的世界霸主,每时每刻都能在它的某个统治区域内见到太阳,真正的日不落帝国。
它的统治者同时也是神圣罗马帝国皇帝,将权力中心放在布拉格。
十六世纪末期哈布斯堡王朝的鲁道夫二世皇帝,热衷于知识。他在艺术,科学(包括占星术和炼金术)和数学方面投入巨大精力,使布拉格成为世界的研究和学术中心。
因此,在这种浓厚的研究氛围中诞生了一个早期的机器人雏形,布拉格石人。

布拉格石人Golem(goh-lem)是一种粘土制成的机器人,在犹太传说中流传甚广。
石人有水火土构成。
在石人额头上刻上emet(希伯来语的“真相”),它就会被赋予生命。
石人虽然能够被真理激活,但却没有独立自主的意识,只能按照人们的指令行事。
幸好石人能令行禁止,因为他的力量非常强大,能够做到很多它的创造者所做不到的事情。
然而,它的绝对服从也有危险性,如不小心给他一些错误指示或因为其他一些意外事件可能会对它的铸造者带来不利影响。
石人力量强大但缺乏智慧。


在某些版本的石人传说中,石人是拉比.犹大为保卫布拉格犹太人想出的办法。
在16世纪中欧的许多地方,布拉格犹太人受到迫害。
拉比.犹大使用卡巴拉的秘密技术建造了石人,并用“真理”驱动它,来保卫布拉格的犹太人民。
但是并不是每个人都支持犹大的行为,担心会因为对生命力量的亵渎而产生意想不到的后果。
最终犹大被迫摧毁了石人,因为石人的蛮力最终导致了许多无辜生命的死亡。拉比.犹大将石人额头上的“emet”的第一个字母擦掉,就变成了“met”(希伯来语“死亡”),石人就被销毁了。

\section{统计学中的石人}
科学家也铸造了自己的石人。
只是这些石人很少有具体的物理形态,都是以计算机代码的形式存在于硅片上。
这些石人就是科学模型。
这些模型通过他们的预测能力以及对直觉的挑战或激发,来影响世界。
对“真理”的探求驱动着这些模型运转,但就像石人或现代机器人一样,科学模型既不是真也不是假,既不是先知也不是骗子。 
相反,它们是为某种目的而设计的结构。 这些结构非常强大,不知疲倦地按照编程逻辑进行计算。


\centering
\begin{tikzpicture}
\tikzset{
	mynode/.style={rectangle,rounded corners,draw=black, top color=white, bottom color=yellow!50,very thick, inner sep=1em, minimum size=3em, text centered},
	myarrow/.style={->, >=latex', shorten >=1pt, thick},
	mylabel/.style={text width=7em, text centered} 
}
\node[mynode] (manufacturer) {工厂}; 
\end{tikzpicture}

